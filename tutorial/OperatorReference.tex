\chapter{Operator reference}
\label{sec:operatorreference}

This chapter describes the built-in operators that come with
\rapidminer. Each operator section is subdivided into several parts:
\begin{enumerate}
\item The group and the icon of the operator.
\item An enumeration of the required input and the generated output
  objects. The input objects are usually consumed by the operator and
  are not part of the output. In some cases this behaviour can be changed by
  using a parameter \parval{keep\_\ldots}. Operators may also receive more
  input objects than required. In that case the unused input objects will be
  appended to the output and can be used by the next operator.
\item The parameters that can be used to configure the
  operator. Ranges and default values are specified. Required parameters
  are indicated by bullets (\textbullet) and optional parameters are
  indicated by an open bullet (\textopenbullet)
\item A list of values that can be logged using the
  \op{ProcessLog} operator (see page \pageref{sec:op:ProcessLog}).
\item If the operator represents a learning scheme, the capabilities
  of the learner are described. The learning capapabilities of most meta
  learning schemes depend on the inner learner.
\item If the operator represents an operator chain a short description of the
  required inner operators is given.
\item A short and a long textual description of the operator.
\end{enumerate}

The reference is divided into sections according to the operator groups known
from the graphical user interface. Within each section operators are
alphabetically listed.

% Notice that some operators extend the functionality of other
% operators. This is indicated by the word \textit{extends}. In this
% case, the operator inherits its parent's input and output classes,
% parameters, and values. Actually all operators extend \op{Operator} or
% \op{OperatorChain} but this is not indicated. 

% Some of the operators are \textit{abstract} and can not be used. They
% only serve as a superclass for other operators that have a common
% behavior or purpose. Abstract classes are set \textit{italic}.

\pagebreak





