%%%%%%%%%%%%%%%%%%%%%%%%%%%%%%%%%%%%%%%%%%%%%%%%%%%%%%%%%%%%%%%%%%%%%%%%%%%%%%%
%%
%%  RapidMiner Operator Reference (alone)
%%
%%  File:     $Id: OperatorReferenceDescription.tex,v 2.2 2007-05-28 14:33:55 ingomierswa Exp $
%%
%%%%%%%%%%%%%%%%%%%%%%%%%%%%%%%%%%%%%%%%%%%%%%%%%%%%%%%%%%%%%%%%%%%%%%%%%%%%%%%
\documentclass{operatorreference}
%\usepackage[ps2pdf,bookmarks=true,bookmarksnumbered=true,colorlinks]{hyperref}

\makeindex

%%  =====  Titelseite  =====
%%{\Huge
\title{
  {\Huge RapidMiner 4.0beta}
}


\address{
  Rapid-I\\
  Mierswa \& Klinkenberg GbR\\
  In der Oeverscheidt 18\\
  44149 Dortmund, Germany\\
  \http rapid-i.com/
}


\begin{document}
\maketitle
\tableofcontents

%%  =====  Eingebundene Abschnitte  =====
%%

\chapter{Notes}

This document contains the description of all operators of \rapidminer as
they can be used within proprietary products after purchasing a license from Rapid-I. 
However, please note the following:

\begin{itemize}
\item The operators that are derived from the Weka machine learning
  library are \emph{not} part of the proprietary version of \rapidminer. If
  you want to use those operators, you have to chose one of the native
  \rapidminer learners instead. Please ask for guidance if it not clear to
  which operators this apply.
\item The functionality of the described operators is guaranteed according to
  the publicly available JUnit tests which are part of the \rapidminer CVS.
\end{itemize}


\chapter{Operator reference}
\label{sec:operatorreference}

This chapter describes the built-in operators that come with
\rapidminer. Each operator section is subdivided into several parts:
\begin{enumerate}
\item The group and the icon of the operator.
\item An enumeration of the required input and the generated output
  objects. The input objects are usually consumed by the operator and
  are not part of the output. In some cases this behaviour can be changed by
  using a parameter \parval{keep\_\ldots}. Operators may also receive more
  input objects than required. In that case the unused input objects will be
  appended to the output and can be used by the next operator.
\item The parameters that can be used to configure the
  operator. Ranges and default values are specified. Required parameters
  are indicated by bullets (\textbullet) and optional parameters are
  indicated by an open bullet (\textopenbullet)
\item A list of values that can be logged using the
  \op{ProcessLog} operator (see page \pageref{sec:op:ProcessLog}).
\item If the operator represents a learning scheme, the capabilities
  of the learner are described. The learning capapabilities of most meta
  learning schemes depend on the inner learner.
\item If the operator represents an operator chain a short description of the
  required inner operators is given.
\item A short and a long textual description of the operator.
\end{enumerate}

The reference is divided into sections according to the operator groups known
from the graphical user interface. Within each section operators are
alphabetically listed.

% Notice that some operators extend the functionality of other
% operators. This is indicated by the word \textit{extends}. In this
% case, the operator inherits its parent's input and output classes,
% parameters, and values. Actually all operators extend \op{Operator} or
% \op{OperatorChain} but this is not indicated. 

% Some of the operators are \textit{abstract} and can not be used. They
% only serve as a superclass for other operators that have a common
% behavior or purpose. Abstract classes are set \textit{italic}.

\pagebreak







\begin{appendix}

\chapter{Regular expressions}
\label{sec:regular_expressions}

Regular expressions are a way to describe a set of strings based on common
characteristics shared by each string in the set. They can be used as a tool
to search, edit or manipulate text or data. Regular expressions range from
being simple to quite complex, but once you understand the basics of how
they're constructed, you'll be able to understand any regular expression.

In \rapidminer several parameters use regular expressions, e.g. for the definition
of the column separators for the ExampleSource operator or for the feature
names of the FeatureNameFilter. This chapter gives an
overview of all regular expression constructs available in \rapidminer. These are
the same as the usual regular expressions available in Java. Further
information can be found at
\begin{center}
\url{http://java.sun.com/docs/books/tutorial/essential/regex/index.html}.
\end{center}



\section{Summary of regular-expression constructs}


\begin{longtable}{|p{3cm}|p{8cm}|}

\hline
 & \\
\textbf{Construct} & \textbf{Matches} \\
 & \\
\hline
\endhead

\hline
\endfoot

\multicolumn{2}{|l|}{\textbf{}}\\
\multicolumn{2}{|l|}{\textbf{Characters}}\\
\hline
x &	The character x \\
$\backslash$$\backslash$ & The backslash character \\
$\backslash$0n & The character with octal value 0n (0 $<$= n $<$= 7) \\
$\backslash$0nn & The character with octal value 0nn (0 $<$= n $<$= 7) \\
$\backslash$0mnn & The character with octal value 0mnn (0 $<$= m $<$= 3, 0 $<$= n $<$= 7) \\
$\backslash$xhh & The character with hexadecimal value 0xhh \\
$\backslash$uhhhh & The character with hexadecimal value 0xhhhh \\
$\backslash$t & The tab character ('$\backslash$u0009') \\
$\backslash$n & The newline (line feed) character ('$\backslash$u000A') \\
$\backslash$r &	The carriage-return character ('$\backslash$u000D') \\
$\backslash$f &	The form-feed character ('$\backslash$u000C') \\
$\backslash$a &	The alert (bell) character ('$\backslash$u0007') \\
$\backslash$e &	The escape character ('$\backslash$u001B') \\
$\backslash$cx & The control character corresponding to x \\
\hline
\multicolumn{2}{|l|}{\textbf{}}\\
\multicolumn{2}{|l|}{\textbf{Character classes}}\\
\hline
{[}abc{{]}}& a, b, or c (simple class) \\
{[}\textasciicircum abc{{]}} & Any character except a, b, or c (negation) \\
{[}a-zA-Z{]} & a through z or A through Z, inclusive (range) \\
{[}a-d{[}m-p{]}{]} & a through d, or m through p: {[}a-dm-p{]} (union) \\
{[}a-z\&\&{[}def{]}{]} & d, e, or f (intersection) \\
{[}a-z\&\&{[}\textasciicircum bc{]}{]} & a through z, except for b and c: {[}ad-z{]} (subtraction) \\
{[}a-z\&\&{[}\textasciicircum m-p{]}{]} & a through z, and not m through p: {[}a-lq-z{]}(subtraction) \\
\hline
\multicolumn{2}{|l|}{\textbf{}}\\
\multicolumn{2}{|l|}{\textbf{Predefined character classes}}\\
\hline
. & Any character (may or may not match line terminators) \\
$\backslash$d &	A digit: {[}0-9{]} \\
$\backslash$D &	A non-digit: {[}\textasciicircum0-9{]} \\
$\backslash$s &	A whitespace character: {[} $\backslash$t$\backslash$n$\backslash$x0B$\backslash$f$\backslash$r{]} \\
$\backslash$S &	A non-whitespace character: {[}\textasciicircum$\backslash$s{]} \\
$\backslash$w &	A word character: {[}a-zA-Z\_0-9{]} \\
$\backslash$W &	A non-word character: {[}\textasciicircum$\backslash$w{]} \\
\hline
\multicolumn{2}{|l|}{\textbf{}}\\
\multicolumn{2}{|l|}{\textbf{POSIX character classes (US-ASCII only)}}\\
\hline
$\backslash$p\{Lower\} & A lower-case alphabetic character: {[}a-z{]} \\
$\backslash$p\{Upper\} & An upper-case alphabetic character: {[}A-Z{]} \\
$\backslash$p\{ASCII\} & All ASCII: {[}$\backslash$x00-$\backslash$x7F{]} \\
$\backslash$p\{Alpha\} & An alphabetic character:{[}$\backslash$p\{Lower\}$\backslash$p\{Upper\}{]} \\
$\backslash$p\{Digit\} & A decimal digit: {[}0-9{]} \\
$\backslash$p\{Alnum\} & An alphanumeric  character:{[}$\backslash$p\{Alpha\}$\backslash$p\{Digit\}{]} \\
$\backslash$p\{Punct\} & Punctuation: One of !"\#\$\%\&'()*+,-./:;$<$=$>$?@{[}$\backslash${]}\textasciicircum\_ $\grave{}$\{|\}$\sim$ \\
$\backslash$p\{Graph\} & A visible character: {[}$\backslash$p\{Alnum\}$\backslash$p\{Punct\}{]} \\
$\backslash$p\{Print\} & A printable character: {[}$\backslash$p\{Graph\}{]} \\
$\backslash$p\{Blank\} & A space or a tab: {[} $\backslash$t{]} \\
$\backslash$p\{Cntrl\} & A control character:  {[}$\backslash$x00-$\backslash$x1F$\backslash$x7F{]} \\
$\backslash$p\{XDigit\} & A hexadecimal digit: {[}0-9a-fA-F{]} \\
$\backslash$p\{Space\} & A whitespace character: {[}  $\backslash$t$\backslash$n$\backslash$x0B$\backslash$f$\backslash$r{]} \\
\hline
\multicolumn{2}{|l|}{\textbf{}}\\
\multicolumn{2}{|l|}{\textbf{Classes for Unicode blocks and categories}}\\
\hline 
$\backslash$p\{InGreek\} & A character in the Greek block (simple block) \\
$\backslash$p\{Lu\} & An uppercase letter (simple category) \\
$\backslash$p\{Sc\} & A currency symbol \\
$\backslash$P\{InGreek\} & Any character except one in the Greek block (negation) \\
{[}$\backslash$p\{L\}\&\&{[}\textasciicircum$\backslash$p\{Lu\}{]}{]} & Any letter except an uppercase letter (subtraction) \\
\hline
\multicolumn{2}{|l|}{\textbf{}}\\
\multicolumn{2}{|l|}{\textbf{Boundary matchers}}\\
\hline 
\textasciicircum & The beginning of a line \\
\$ & The end of a line \\
$\backslash$b &	A word boundary \\
$\backslash$B &	A non-word boundary \\
$\backslash$A &	The beginning of the input \\
$\backslash$G &	The end of the previous match \\
$\backslash$Z &	The end of the input but for the final terminator, if any \\
$\backslash$z &	The end of the input \\
\hline
\multicolumn{2}{|l|}{\textbf{}}\\
\multicolumn{2}{|l|}{\textbf{Greedy quantifiers}}\\
\hline  
X? & X, once or not at all \\
X* & X, zero or more times \\
X+ & X, one or more times \\
X\{n\} & X, exactly n times \\
X\{n,\} & X, at least n times \\
X\{n,m\} & X, at least n but not more than m times \\
\hline
\multicolumn{2}{|l|}{\textbf{}}\\
\multicolumn{2}{|l|}{\textbf{Reluctant quantifiers}}\\
\hline   
X?? &	X, once or not at all \\
X*? &	X, zero or more times \\
X+? &	X, one or more times \\
X\{n\}? &	X, exactly n times \\
X\{n,\}? & X, at least n times \\
X\{n,m\}? & X, at least n but not more than m times \\
\hline
\multicolumn{2}{|l|}{\textbf{Logical operators}}\\
\hline    
XY & X followed by Y \\
X|Y & Either X or Y \\
(X) & X, as a capturing group \\
\hline
\multicolumn{2}{|l|}{\textbf{}}\\
\multicolumn{2}{|l|}{\textbf{Back references}}\\
\hline 
$\backslash$n & Whatever the $n$-th capturing group matched \\
\hline
\multicolumn{2}{|l|}{\textbf{}}\\
\multicolumn{2}{|l|}{\textbf{Quotation}}\\
\hline    
$\backslash$ & Nothing, but quotes the following character \\
$\backslash$Q & Nothing, but quotes all characters until $\backslash$E \\
$\backslash$E & Nothing, but ends quoting started by $\backslash$Q \\
\hline
\multicolumn{2}{|l|}{\textbf{}}\\
\multicolumn{2}{|l|}{\textbf{Special constructs (non-capturing)}}\\
\hline    
(?:X) & X, as a non-capturing group \\
(?idmsux-idmsux) & Nothing, but turns match flags on - off \\
(?idmsux-idmsux:X) & X, as a non-capturing group with the given flags on - off \\
(?=X) & X, via zero-width positive lookahead \\
(?!X) & X, via zero-width negative lookahead \\
(?$<$=X) & X, via zero-width positive lookbehind \\
(?$<$!X) & X, via zero-width negative lookbehind \\
(?$>$X) & X, as an independent, non-capturing group \\
\hline
\end{longtable}


\end{appendix}

%%  =====  Literaturverzeichnis  =====
%%
%\bibliographystyle{plain} 
%\bibliography{RapidMinerTutorial}

\printindex

\end{document}
