\section{Cluster algorithms}

In case that the example set has no label attribute, unsupervised
learning methods can be applied. Clustering algorithms segment the
example space according to a given distance measure.



\absoperator{Clusterer}

\newcommand{\clusterio}{
\begin{opin}
\item[ExampleSet:] the example set to be clustered
\end{opin}

\begin{opout}
\item[Model:] the learned cluster model
\item[ExampleSet:] the clustered example set
\end{opout}
}
\clusterio

\begin{parameters}
\optpar[cluster\_file] name of the file the \op{Clusterer} writes the learned cluster model into
\end{parameters}

\opdescr \op{Clusterer} is the abstract superclass of all clusterer
operators. Values are assigned to the cluster attributes of the examples.



\operator[Clusterer]{WekaClusterer}

\clusterio

\begin{parameters}
\reqpar[weka\_clusterer\_name] fully qualified WEKA classname of the WEKA clusterer to be used, 
e.g. \texttt{weka.clusterers.EM} for the expectation maximization approach
\optpar All parameters from the group \para{weka\_parameters} are
assumed to be key-value pairs that are passed to the classifier. The
leading dash in front of the parameter name must not be part of the key!
\optpar[cluster\_file] name of the file the \op{Clusterer} writes the learned model into
\end{parameters}

\opdescr This operator can wrap all clusterers from the Weka clusterers
package. The clusterer type can be selected by a parameter. 
See the Weka javadoc for descriptions of the clusteres types 
that are available (\url{http://www.cs.waikato.ac.nz/ml/weka/}).



\operator[OperatorChain]{ClusterWrapper}

\begin{opin}
\item[ExampleSet] a clustered example set
\end{opin}

\begin{opout}
\item none
\end{opout}

\opdescr This operator applies its inner operators once for each cluster.
